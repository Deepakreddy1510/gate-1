\documentclass[journal,12pt,twocolumn]{IEEEtran}
\usepackage{amsmath,amssymb,amsfonts,amsthm}
\usepackage{txfonts}
\usepackage{tkz-euclide}
\usepackage{listings}
\usepackage{gvv}
\usepackage[latin1]{inputenc}
\usepackage{adjustbox}
\usepackage{array}
\usepackage{tabularx}
\usepackage{pgf}
\usepackage{lmodern}
\usepackage{circuitikz}
\usepackage{tikz}
\usepackage{graphicx}
\begin{document}
\bibliographystyle{IEEEtran}

\vspace{3cm}

\title{}
\author{EE23BTECH11047 - Deepakreddy P
}
\maketitle
\newpage
\bigskip

\noindent \textbf{44} \quad The switch $S_1$ was closed and $S_2$ was open for a long time. At t=0,switch $S_1$ is opened and $S_2$ is closed,simultaneously. The value of $i_c(0^{+})$, in amperes, is  \hfill (GATE EC 44)

\begin{figure}[ht]
  \centering
  \begin{adjustbox}{width=\columnwidth}
      \begin{circuitikz}[american]
   \draw (0,0) to [isource, l=1A] (0,4) ;
   \draw (0,0) to [short] (3,0) to [C = 0.01F] (3,4) to [short] (6,4) to [R = 100 $\Omega$] (6,2) to [L = 1H] (6,0) to [short] (9,0) to [R = 25 $\Omega$] (9,4) to [short] (6,4) ;
   \draw (3,0) to [short] (6,0) ;
   \draw (0,4) to [ospst = $S_1$] ++(3,0); 
   \draw (7.5,4) to [cspst = $S_2$] ++(0,-2);
   \draw (7.5,2) to [short] (6,2) ;
   \draw (3,4) to [short ,i = $i_c$] (3,3);
   
\end{circuitikz}

  \end{adjustbox}
  \caption{Circuit 1}
\end{figure}

\solution
\\
1) Switch $S_1$ was closed and $S_2$ was open 
\begin{figure}[ht]
  \centering
  \begin{adjustbox}{width=\columnwidth}
      \begin{circuitikz}[american]
   \draw (0,0) to [isource, l=1A] (0,4) ;
   \draw (0,0) to [short] (3,0);
   \draw (0,4) to [short] (9,4) to [R=25$\Omega$] (9,0) to [short] (3,0);
   \draw (6,4) to [R = 100$\Omega$] (6,0);
   \draw (3,4) -- ++(0,-1.5)
   to [open, v = $V_c(0^{-})$, o-o] ++(0,-1.5) -- ++(0,-1);
   \draw (6,1) to [short ,i = $i_L \brak{0^-}$] (6,0);
   
\end{circuitikz}

  \end{adjustbox}
  \caption{Circuit 2}
\end{figure}

\begin{align}
    R_{eff} &= 5 \Omega \\
    i_{L} \brak{0^-} &= \frac{\frac{25}{s}}{125}  = \frac{0.2}{s}\\
    i_{L} \brak{0^-} &= i_{L} \brak{0^+}\\
    V_c \brak{0^-} &= 20V
\end{align}

\bigskip

2) Switch $S_1$ is open and $S_2$ was closed\\

At $t=0^{+}$ The capacitor is charged. Thus, it acts as a voltage source.The inductor acts as current source.\\

\bigskip

\begin{figure}[ht]
  \centering
  \begin{adjustbox}{width=\columnwidth}
      \begin{circuitikz}[american]
   \draw (3,4) to [C=$\frac{1}{sC} $ ](3,2) to [V=$\frac{V_c \brak{0^-}}{s}$ ] (3,0) ;
   \draw (3,4) to [short] (9,4);
   \draw (8,0) to [V=$L i_L \brak{0^-}$] (8,2) ;
   \draw (8,4) to [L = $Ls$](8,2) ;
   \draw (8,4) to [short] (10,4) to [R = 25 $\Omega$] (10,0) to [short] (8,0);
   \draw (3,0) to [short] (8,0) ;
   \draw (3,2) to [short, i = $i_{c} \brak{s}$] (3,1.5);
   \draw (10,1) to [short ] (10,0);
\end{circuitikz}

  \end{adjustbox}
  \caption{Circuit 3}
\end{figure}

\begin{align}
    i_c \brak{0^+} = - \frac{1}{s}
\end{align}

\bigskip
Taking Inverse Laplace Transform\\

\begin{align}
    i_c \brak{0^+} = -1A
\end{align}

\begin{table}[ht]
    \begin{adjustbox}{width=\columnwidth}
       \setlength{\arrayrulewidth}{0.3mm}
\setlength{\tabcolsep}{20pt}
\renewcommand{\arraystretch}{1.3}



\begin{tabular}{|c|c|c|}
\hline

Parameter& Description & Remarks\\
\hline
$V_c\brak{0^{-}}$ & Voltage across capacitor in case 1 & 20V\\
\hline
$i_L \brak{0^-}$ & current across inductor in case 1  & $\frac{0.2}{s} $  \\
\hline
$i_L \brak{0^+}$ & current across inductor in case 2 & $\frac{0.2}{s} $\\
\hline
$C$ & Capacitance & 0.01F\\
\hline

\end{tabular}


    \end{adjustbox}
    \caption{Parameters}
    \label{tab:Gate.ec.44.1}

\end{table}
\end{document}



