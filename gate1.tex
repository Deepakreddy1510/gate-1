\documentclass[journal,12pt,twocolumn]{IEEEtran}
\usepackage{amsmath,amssymb,amsfonts,amsthm}
\usepackage{txfonts}
\usepackage{tkz-euclide}
\usepackage{listings}
\usepackage{gvv}
\usepackage[latin1]{inputenc}
\usepackage{adjustbox}
\usepackage{array}
\usepackage{tabularx}
\usepackage{pgf}
\usepackage{lmodern}
\usepackage{circuitikz}
\usepackage{tikz}
\usepackage{graphicx}
\begin{document}
\bibliographystyle{IEEEtran}

\vspace{3cm}

\title{}
\author{EE23BTECH11047 - Deepakreddy P
}
\maketitle
\newpage
\bigskip

\noindent \textbf{44} \quad The switch $S_1$ was closed and $S_2$ was open for a long time. At t=0,switch $S_1$ is opened and $S_2$ is closed,simultaneously. The value of $i_c(0^{+})$, in amperes, is  \hfill (GATE EC 44)

\begin{figure}[ht]
  \centering
  \begin{adjustbox}{width=\columnwidth}
      \input{figs/ckt1}
  \end{adjustbox}
  \caption{Circuit 1}
\end{figure}

\solution
\\
1) Switch $S_1$ was closed and $S_2$ was open 
\begin{figure}[ht]
  \centering
  \begin{adjustbox}{width=\columnwidth}
      \input{figs/ckt2}
  \end{adjustbox}
  \caption{Circuit 2}
\end{figure}

\begin{align}
    R_{eff} &= 20 \Omega \\
    i_{L} \brak{0^-} &= \frac{25}{125} A = 0.2A\\
    i_{L} \brak{0^-} &= i_{L} \brak{0^+}\\
\end{align}
Apply KVL 
\begin{align}
    -V_c \brak{0^-} + 20 \brak{1} &= 0\\
    V_c \brak{0^-} &= 20V
\end{align}

\bigskip

2) Switch $S_1$ is open and $S_2$ was closed\\

At $t=0^{+}$ The capacitor is charged. Thus, it acts as a voltage source.The inductor acts as the current source.

\bigskip

\begin{figure}[ht]
  \centering
  \begin{adjustbox}{width=\columnwidth}
      \begin{circuitikz}[american]
   \draw (3,4) to [C=$\frac{1}{sC} $ ](3,2) to [V=$\frac{V_c \brak{0^-}}{s}$ ] (3,0) ;
   \draw (3,4) to [short] (6,4) to [L = $Ls$](6,2) to [V=$-i_L \brak{0^-}$] (6,0) to [short] (9,0) to [R = 25 $\Omega$] (9,4) to [short] (6,4) ;
   \draw (3,0) to [short] (6,0) ;
   \draw (3,2) to [short, i = $i_{c} \brak{0^+}$] (3,1.5);
   \draw (9,1) to [short ,i = $\frac{V_c \brak{0^-}}{25s}$] (9,0);
\end{circuitikz}

  \end{adjustbox}
  \caption{Circuit 3}
\end{figure}

\bigskip


\begin{align}
    i_c \brak {0^+} &= - \brak {i_L \brak{0^-} + \frac{V_c \brak{0^-}}{25s}}\\
    i_c \brak{0^+} &= -\brak{0.2 + \frac{0.8}{s}}
\end{align}
Taking Inverse Laplace Transform
\begin{align}
      i_c \brak{0^+} &= -0.2 \delta \brak{t} - 0.8u \brak{t}\\
      &=-1A
\end{align}

\begin{table}[ht]
    \begin{adjustbox}{width=\columnwidth}
       \setlength{\arrayrulewidth}{0.3mm}
\setlength{\tabcolsep}{20pt}
\renewcommand{\arraystretch}{1.3}



\begin{tabular}{|c|c|c|}
\hline

Parameter& Description & Remarks\\
\hline
$V_c\brak{0^{-}}$ & Voltage across capacitor in case 1 & 20V\\
\hline
$i_L \brak{0^-}$ & current across inductor in case 1  & $\frac{0.2}{s} $  \\
\hline
$i_L \brak{0^+}$ & current across inductor in case 2 & $\frac{0.2}{s} $\\
\hline
$C$ & Capacitance & 0.01F\\
\hline
$L$ & Inductance & 1H\\
\hline

\end{tabular}


    \end{adjustbox}
    \caption{Parameters}
    \label{tab:Gate.ec.44.1}

\end{table}

\newpage

\begin{figure}[ht]
   \centering
   \includegraphics[width=1.2\columnwidth]{figs/ckt2.png}
   \caption{Plot of $V_c(0^-)$ vs time}
\end{figure}

\end{document}



